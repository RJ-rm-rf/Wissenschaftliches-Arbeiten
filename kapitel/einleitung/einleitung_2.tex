\section{Einleitung}
\subsection{Einführung}
Heutzutage begegnen Menschen \ac{KI}-Systemen, zumeist auch unbewusst, in vielen Bereichen des alltäglichen Lebens. Sie gewinnt in Unternehmen, Verwaltungen und dem alltäglichen
Leben zunehmend an Bedeutung. 
Mittlerweile hat die \ac{KI} in allen Lebensbereichen wie Freizeit, Beruf, Schule, Wirtschaft, Politik eine enorme mitgestaltende Bedeutung erhalten. \ac{KI} dominiert 
längst das Leben der Menschen, ohn dass es vielen bewusst ist. Sie nutzen Smartphones, die mit ihnen sprechen, tragen Armbanduhren, die ihre Gesundheitsdaten aufzeichnen und arbeiten nach Arbeitsabläufen, die 
sich automatisch organisieren. In Zukunft werden Sie Autos, Flugzeuge und Drohnen einsetzen, die sich selber steuern. Diese sind einige wenige Beispiele einer vernetzten Welt intelligenter 
Systeme, welche aufzeigen, wie der Alltag der Menschen von \ac{KI}-Systemen bestimmt ist.~\footcite[\vglf][\pagef 7]{Mainzer.2019}
\subsection{Problemstellung}
Die Digitalisierung und Entwciklung auf dem Gebiet der Computertechnik und des maschinellen Lernens haben in den vergangenen Jahren zu rasanten Fortschritten geführt. Die Entwicklung
durchzieht alles Bereiche des Gesellschaftlichen Lebens, im privaten sowie beruflichen und schulischen, im wissenschaftlichen, politischen und rechtlichen Bereich. Alle gesellschaftlichen
Schichten sind ausnahmslos davon betroffen.~\footcite[\vglf][\pagef 7]{Lenzen.2020}
In Anbetracht dessen führen die immer fortschreitende Entwicklung der KI zu immer neuen Möglichkeiten, welche sich mit großen Hoffnungen für die Gesellschaft verbunden sind, wie z. B. im Gesundheitsweisen, 
oder beim autonomen Fahren. 
Auf der anderen Seite befürchten viele Menschen allerdings einen ehrehblichen negativen Einfluss auf den Arbeitsmarkt, oder den Verlust der Entscheidungsfreiheit~\footcite[\vglf][\pagef 5]{Wittpahl.2018}.
Die wachsende Rolle der \ac{KI} im gesellschaftlichen Bereich mit Ihren Vor- und Nachteilen fürht zu neuen Herausforderungen der sich alle Akteure, sowieo Profiteure, kritisch 
gegenüberstellen und auseinander setzen müssen, um eine adäquate Nutzung der \ac{KI} zu gewährleisten. Zielsetzung dieser Arbeit ist es, herauszustellen, welche Möglichkeiten
der Einsatz der \ac{KI} für die Gesellschaft bietet und wo die Gefahren liegen. Daraus werden zukünftige Herausforderungen abgeleitet, denn wie jede Technolgie ist auch die \ac{KI}
kein ausweichliches Schicksal. Die Gesellschaft gibt ihr Ihre Form. 

\subsection{Aufbau der Arbeit}
Diese Arbeit befasst sich diese Arbeit mit dem Thema \enquote{Die Chancen und Risiken beim Einsatz von künstlicher Intelligenz in der modernen Gesellschaft und die damit verbundenen Herausforderungen}.
In dieser werden die Chancen und Risiken des Ensatz von \ac{KI} erörtert und die damit verbunden zukünftigen Herausforderungen für die Menschen beleuchtet. Die Arbeit beginnt
mit einem Einblick in den Grundlagenbereich der \ac{KI}.
Im Hauptteil und zugleich Kern der ARbeit geht es um die Nutzung der \ac{KI} in der Gesellschaft. Hier werden die Chancen und Risiken des Einsatzes von \ac{KI} dargestellt und
hinsichtlich ihrer Vorteile und Nachteile analysiert. Im Anschluss daran wird werden die Herausforderungen abgeleitet und hinsichtlich DUYGU schreibt dat gleich.