\section{Einleitung}
\subsection{Einführung}
Heutzutage begegnen Menschen \ac{KI}-Systemen, zumeist auch unbewusst, in vielen Bereichen des alltäglichen Lebens. Sie gewinnen in Unternehmen, Verwaltungen und dem alltäglichen
Leben zunehmend an Bedeutung. 
Mittlerweile hat die \ac{KI} in allen Lebensbereichen wie Freizeit, Beruf, Schule, Wirtschaft, Politik eine enorme mitgestaltende Bedeutung erhalten. \ac{KI} dominiert 
längst das Leben der Menschen, ohne dass es vielen bewusst ist. Sie nutzen Smartphones, die mit ihnen sprechen, tragen Armbanduhren, die ihre Gesundheitsdaten aufzeichnen und arbeiten nach Arbeitsabläufen, die 
sich automatisch organisieren. In Zukunft werden sie Autos, Flugzeuge und Drohnen einsetzen, die sich selber steuern. Diese sind einige wenige Beispiele einer vernetzten Welt intelligenter 
Systeme, welche aufzeigen, wie der Alltag der Menschen von \ac{KI}-Systemen bestimmt ist.~\footcite[\vglf][\pagef 7]{Mainzer.2019}
\subsection{Problemstellung}
Die Digitalisierung und Entwicklung auf dem Gebiet der Computertechnik und des maschinellen Lernens haben in den vergangenen Jahren zu rasanten Fortschritten geführt. Die Entwicklung
durchzieht alle Bereiche des gesellschaftlichen Lebens: im privaten, beruflichen, schulischen, wissenschaftlichen, politischen und rechtlichen Bereich. Alle gesellschaftlichen
Schichten sind ausnahmslos davon betroffen.~\footcite[\vglf][\pagef 7]{Lenzen.2020}
In Anbetracht dessen führen die immer fortschreitenderen Entwicklungen der KI zu immer neueren Möglichkeiten, welche mit großen Hoffnungen für die Gesellschaft verbunden sind, wie z. B. im Gesundheitsweisen 
oder beim autonomen Fahren. 
Auf der anderen Seite befürchten viele Menschen allerdings einen erheblichen negativen Einfluss bspw. auf dem Arbeitsmarkt oder den Verlust der Entscheidungsfreiheit~\footcite[\vglf][\pagef 5]{Wittpahl.2018}.
Die wachsende Rolle der \ac{KI} im gesellschaftlichen Bereich mit ihren Vor- und Nachteilen führt zu neuen Herausforderungen, denen sich alle Akteure sowie Profiteure kritisch 
gegenüberstellen und auseinandersetzen müssen, um eine adäquate Nutzung der \ac{KI} zu gewährleisten. Zielsetzung dieser Arbeit ist es, herauszustellen, welche Chancen
der Einsatz der \ac{KI} für die Gesellschaft bietet und wo ihre Risiken im Einsatz und Umgang mit ihr liegen. Daraus sollen zukünftige Herausforderungen für die Menschheit abgeleitet und ggf. Handlungsempfehlungen formuliert werden.

\subsection{Aufbau der Arbeit}
Vor dem Hintergrund der zuvor dargestellten Problematik, befasst sich die vorliegende Arbeit mit dem Thema: \enquote{Die Chancen und Risiken beim Einsatz von künstlicher Intelligenz in der modernen Gesellschaft und die damit verbundenen Herausforderungen}.
Im Rahmen dieser Arbeit werden die Chancen und Risiken des Ensatzes von \ac{KI} erörtert und die damit verbunden zukünftigen Herausforderungen für die Menschen beleuchtet.
Dabei beginnt die Arbeit mit einem Einblick in den Grundlagenbereich der \ac{KI}, um die theoretischen Voraussetzungen für die Weiterarbeit zu schaffen.
Im Hauptteil und zugleich Kern der Arbeit wird es um die Nutzung der \ac{KI} in der Gesellschaft gehen. Hier werden die Chancen und Risiken des Einsatzes von \ac{KI} dargestellt und
hinsichtlich ihrer Vorteile und Nachteile analysiert. Im Anschluss daran werden die zukünftigen Herausforderungen für die Gesellschaft abgeleitet und die Bedeutung der Beschäftigung mit der KI aufgezeigt. Es wird u.a. der Frage nachgegangen, 
warum es heutzutage notwendig ist, sich gesellschaftlich, politisch, beruflich, rechtlich und ethisch mit der KI zu befassen. 
   