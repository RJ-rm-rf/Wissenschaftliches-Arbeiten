\newpage
\section{Fazit}\label{Fazit}
Die vorliegende Arbeit hat sich mit der Thematik \enquote{Die Chancen und Risiken beim Einsatz von künstlicher Intelligenz in der modernen Gesellschaft und die damit verbundenen Herausforderungen}
beschäftigt. Dabei wurde herausgearbeitet, welche Chancen und Risiken der Einsatz von \ac{KI} in unserer heutigen Gesellschaft umfasst und welche zukünftigen Herausforderungen
auf die Menschheit zukommen.
Bezüglich der Chancen des Einsatzes von \ac{KI} lässt sich resümierend feststellen, dass es vielfältige Möglichkeiten gibt. Intelligente Assistenten werden die Menschen immer mehr in ihrem
alltäglichen Leben begleiten. Der Mensch wird eine Kooperation mit den intelligenten Assistenten eingehen, in der die jeweiligen Stärken optimal ausgenutzt werden. 
Im gleichen Zug werden soziale Roboter einen immer höheren Anteil in unserem Leben einnehmen. Sie übernehmen Arbeiten, in denen sie besser sind als Menschen oder die Menschen
ungerne verrichten. Sie werden zum normalen Bild im modernen Alltag.
Weiterhin wird der Verkehrsfluss optimiert, durch eine \ac{KI} gesteuerte Verkehrsinfrastruktur mit dem wesentlichen Vorteil, dass es weniger Verkehrstote gibt und die Sicherheit
im Straßenverkehr deutlich erhöht wird. Das Ganze wird auch einen positiven Effekt auf die Umwelt haben, nämlichc durch eine andauernde Optimierung wird bspw. die Reichweite von Elektrofahrzeugen bei sinkendem Energieverbrauch erhöht.
Durch die hervorragende Mustererkennung wird die \ac{KI} immer mehr im Gesundheitweisen, in der medizinischen Forschung und bei der Behandlung von Krankheiten eingesetzt. 
Durch die Zusammenarbeit der \ac{KI} und dem ärztlichen Personal können präzisere Diagnosen erstellt und fehlerhafte Diagnosen vermieden werden.
Hinsichtlich der Risiken beim Einstz von \ac{KI} lässt sich feststellen, dass diese einen großen Einfluss auf den Arbeitsmarkt haben wird. Es besteht die Gefahr, dass auf kurze Sicht,
besonders Arbeitstellen in Gefahr sind, die eine niedrige Qualifikation erfordern. Auf lange Sicht, wird sich die \ac{KI} auch auf kognitive Berufe erstrecken. Die Gesellschaft wird nach Möglichkeiten
suchen müssen, ein Leben ohne diese Berufsfelder zu führen.
Auch besteht die Gefahr, dass \ac{KI} zur Überwachung, sozialer Kontrolle und Diskriminierung eingesetzt werden kann. Das chinesische Sozialkredit-System ist ein modernes Beispiel hierfür. Auch findet soziale Kontrolle, in einer abgeschwächteren Version,
auch in unserem Alltag statt, indem wir an uns angepasste Werbung erhalten oder in Städten und an Flughäfen per Gesichtserkennung überwacht werden.
Ein großer Schwerpunkt beim Einsatz von \ac{KI} sind autonome Waffensysteme. Eine wesentliche und ungeklärte Rolle ist die Frage der Verantwortung bei der Verwendung von autonomen Waffensystemen. Ferner besteht die Gefahr, dass die Hemmschwelle zur Konflikteskalation beim Einsatz von autonomen Waffensystemen sinkt, da es keine menschliche Beteiligung gibt.
Aufgrund der enormen Datenverarbeitungskapazitäten können solche Situationen schnell eskalieren und durch den Menschen nicht mehr kontrollierbar werden, was verheerende
Auswirkungen haben könnte.
Die aus den Chancen und Risiken resultierenden Herausforderungen, machen deutlich, dass die Gesellschaft sich vorallem mit den ethischen Dimensionen des Einsatzes von \ac{KI} auseinandersetzen und Regeln aufstellen müssen.
Durch die enorm hohe Verabeitungsrate werden die getroffenen Entscheidungen einer \ac{KI} für den Menschen nicht mehr nachvollziehbar, welches besonders in Bereichen sehr gefährlich ist, in denen
Menschen betroffen sind. 
Desweiteren muss die Privatsphäre der Menschen geschützt werden, damit solche \ac{KI} nicht alles verarbeiten und darauf Schlussfolgerungen ziehen, die für den einzelnen problembehaftet sind. Ein erster
Schritt erfolgte in der Verabschiedung der \ac{DSGVO} oder in der Evaluierung globaler Standards für den Umgang.
Alles in allem lässt sich sagen, dass der Erfolg von \ac{KI} ein ethisches Regelwerk vorraussetzt. Wenn dies gegeben ist, lassen sich die Stärken von \ac{KI} entfalten und der Missbrauch
verhindern. 