\newpage
\section{Fazit}\label{Fazit}
Die vorliegende Arbeit hat sich mit der Thematik \enquote{Die Chancen und Risiken beim Einsatz von künstlicher Intelligenz in der modernen Gesellschaft und die damit verbundenen Herausforderungen}
beschäftigt. Dabei wurde herausgearbeitet, welche Chancen und Risiken der Einsatz von \ac{KI} in unserer heutigen Gesellschaft umfasst und welche zukünftigen Herausforderungen
auf die Menschheit zukommen.
Bezüglich der Chancen des Einsatzes von \ac{KI} lässt sich resümierend feststellen, dass es vielfältige Möglichkeiten gibt. Intelligente Assistenten werden die Menschen immer mehr in ihrem
alltäglichen Leben begleiten. Der Mensch wird eine Kooperation mit den intelligenten Assistenten eingehen, in der die jeweiligen Stärken optimal ausgenutzt werden. 
Im gleichen Zug werden soziale Roboter einen immer höheren Anteil in unserem Leben einnehmen. Sie übernehmen Arbeiten, in dennen Sie besser sind als Menschen, oder die Menschen
ungerne Verrichten. Sie werden zum normales Bild im modernen Alltag.
Weiterhin wird der Verkehrsfluss optimiert, durch eine \ac{KI} gesteuerte Verkehrsinfrastruktur, mit dem wesentlichen Vorteil, dass es weniger Verkehrstote gibt und die Sicherheit
im Straßenverkehr deutlich erhöht wird. Das ganze wird auch einen positiven Effekt auf die Umwelt haben, durch andaurende Optimierung z. B. die Reichweite von Elektrofahrzeugen 
erhört wird, bei sinkendem Energieverbrauch.
Durch die hervoragende Musterkrennung wird \ac{KI} auch immer mehr im Gesundheitweisen eingesetzt, in der medizinischen Forschung und bei der Behandlung von Krankheiten. 
Dabei werden sie mit menschlichen Ärzten zusammenarbeiten.
Hinsichtlich der Risiken beim Einstz von \ac{KI} lässt sich feststellen, dass diese einen großen Einfluss auf den Arbeitsmarkt haben wird. Es besteht die Gefahr, dass auf kurze Sicht,
besodners Arbeitstellen in Gefahr sind, die eine niedrige Qualifikation erfordern. Auf lange Sicht, wird sich \ac{KI} auch auf kogntive Berufe erstrecken. Die Gesellschaft wird nach Möglichkeiten
suchen müssen, ein Leben ohne Arbeit zu führen.
Auch besteht die Gefahr das \ac{KI} zur Überwachung, soziale Kontrolle und Diskriminierung eingesetzt wird. Das chinesische Sozialkredit-System ist ein modernes Beispiel dafür. Auch findet es
in unserem Alltag statt, indem wir maßgeschneiderte Werbung erhalten oder in Städten oder an Flughäfen per Gesichtserkennung überwacht werden.
Ein ganz großes Punkt beim Einsatz von \ac{KI} sind autonome Waffensysteme. Dabei stellt sich die Frage der Verantwortung beim Einsatz. Vorallem singt die Hemmschwelle zur Konflikteskalation, da es keine Menschliche beteiligung gibt.
Aufgrund der enormen Datenverarbeitungskapazitäten können solche Situationen schnell eskalieren und durch den Menschen nicht mehr kontrollierbar werden, was eine verherrende
Auswirkungen haben könnte.
Die darauf resultierenden Herausforderungen, machen deutlich, dass wir uns vorallem mit den ethischen Dimensionen des Einsatzes von \ac{KI} auseinander setzen und Reglen aufstellen müssen.
Durch die enorm hohe Verabeitungsrate werden die getroffener Entscheidungen einer \ac{KI} für den Menschen nicht mehr nachvollziehbar, was besonders in Bereichen sehr gefährlich ist, in dennen
Menschen davon betroffen sind. 
Auch muss die Privatsphere der Menschen beschützt werden, damit solche \ac{KI} nicht alles verarbeiten und darauf Schlussfolgerungen ziehen, die für den einzelnen Problembehaftet sind. Ein erster
Schritt erfolgte in der verabschiedung der \ac{DSGVO} oder in der evualuierung globaler Standards für den Umgang.
Alles in allem lässt sich sagen, dass der Erfolg von \ac{KI} ein ethisches Regelwerk vorraussetzt. Wenn dies gegeben ist, lassen sich die Stärken von \ac{KI} entfalten und der Missbrauch
verhindern. 