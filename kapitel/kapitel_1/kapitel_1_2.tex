\newpage
\section{Theoretische Grundlagen} \label{grundlange}

\subsection{Künstliche Intelligenz}

Die \ac{KI} ist ein Teilgebiet der Informatik und der Ingenieurswissenschaften. Innerhalb der Gesellschaft wird \ac{KI} als eine Simulation intelligenten 
menschlichen Denkens und Handelns aufgefasst\footcite[][\pagef 2]{Mainzer.2019}. 
Die KI-Forschung hat sich allerdings schon seit längerem von der Imitation menschlicher Intelligenz emanzipiert. 
Die Wissenschaft hat erkannt, dass es bessere Problemlässungsansätze gibt, als die Imitations des menschlichen Gehirns.
Vielmehr arbeitet die Forschung daran, Regeln und Prinzipien zu finden, die es einem Computer erlauben die kogntiven Prozesse eines Menschen durch Berechnungsprozesse 
nachzubilden~\footcite[\vglf][\pagef 11]{Lenzen.2020}.

Moderne KI-Projekte sind eine Kooperation der Forschungsgebiet der Informatik, der Ingenieurskünste, der Mathematik, der Psychologie, der Biologie, der Linguistik,
der Neurowissenschaften, der Philosophie und der Ethnologie.
Die daraus hervorgehenden System können zum Beispiel Sätze analysieren und Fragen beantworten, zum Beispiel zum Inhalt eines Textes, indem sie Sprache verschriftlichen.
Auch sind sie in der Lage Bilder zu erkennen, dieser zu analysieren und auf dieser Grundlage eigenständige Werke zu erschaffen.
Prominente Beispiele für solche System sind \enquote{ChatGPT} und \enquote{DALL·E2}.
Den Systemen stehen sehr große Mengen an Daten zur Verfügungen, die durch sie auf erkennbar Muster durchsucht werden. So können Sie die Auswirkung einer Entscheidung im vorraus 
berechnenen und die Menschen so bei Entscheidungen unterstützen~\footcite[\vglf][]{Lenzen.2020}.

\ac{KI} kann in \enquote{starke} und \enquote{schwache} \ac{KI} kategorisiert werden. Dabei wäre eine starke \ac{KI} etwa eine Maschine,
welche eine ähnliche Intelligenz und Flexibilität wie ein Mensch besitzt. Eine schwache \ac{KI} hingegen ist ein System, welches nur eine Aufgabe besitzt.
Diese Kategorie bildet auch den größten Teil der KI-Forschung und Produktentwicklung~\footcite[\vglf][]{Lenzen.2020}.

Bei \ac{KI}-Systemen im Allgemeinen besteht aber die Herausforderung zu entscheiden, ab wann diese als intelligent gelten, da es keine klare Definition des Wortes \enquote{Intelligenz} gibt.
Oft wird der Mensch als Maßstab für Definitionsansätze verwendet. Die \ac{KI}, auf heutigen Stand, übertrifft den Menschen jedoch bei weitem auf einem speziellen Gebiet wie z.B. Schach spielen,
oder, oder wie oben bereits erwähnt, in große Datenmengen, im folgenden nur noch als Big Data bezeichnet, nach Mustern zu durchsuchen, da sie anders funktionieren als der menschliche Verstand.
Konsens bei dem Verständnis von Intelligenz ist aber, dass sie auf Flexibilität und Lernen beruht und mit der Fähigkeit, auf wechselnde Anforderungen zu reagieren und die eigene 
Verhaltensweise erfahrungsbasiert anzupassen~\footcite[\vglf][\pagef 14]{Lenzen.2020}. 

\begin{figure}[H]
    \caption{Was ist Deep Learning?}
    \includegraphics[width=0.9\textwidth]{deepL}
    \label{fig:deepL}
    \\
    \cite[Quelle: Vgl.][]{DeepLearning}
\end{figure}

Damit \ac{KI}-Systeme lernen, wird das sog. \enquote{maschinelle Lernen} eingesetzt. Wie in Abbildung \ref{fig:deepL} dargestellt ist, basiert dieses Verfahren auf Algorithmen die von Daten lernen können. Tiefergreifend kommt dabei das 
sog \enquote{Deep Learning}-Verfahren zum Einsatz welches auf künstlichen neuronalen Netzen basiert~\footcite[\vglf][]{Lenzen.2020}.
Im Zuge der Digitalisierung wird unsere analoge Welt für solche informationsverarbeitenden Systeme in Form von Big Data lesbar gemacht und als Lernquelle zu Verfügung gestellt.
Trotz allem bleiben die \ac{KI}-Systeme hoch spezialisiert und können sich nicht mit der flexiblen Intelligenz der Menschen messen. Um diese Hürde zu überwinden, nähert sich die 
aktuelle \ac{KI}-Forschung wieder an die Neurowissenschaft und der menschlichen Kognition an~\footcite[\vglf][\pagef 18]{Lenzen.2020}.

\subsection{Maschinelles lernen}

Eine manuelle Erstellung von Regeln und Wissenpräsentationen, für die Verarbeitung durch \ac{KI}-Systeme, stellt einen hohen Aufwand mit nur einem begrenzten Nutzen dar~\footcite[\vglf][\pagef 4]{Matzka.2021}.
Um diesen Vorgang zu optimieren, werden nach Algorithmen und Techniken geforscht, die es \ac{KI}-Systemen ermöglichen selbständig allgemeingültige Regeln zu abstrahieren, in dem es selbständig Muster, aus einem ihm zu Verfügungstehenden
Datensatz, erkennt. Dies soll die Systeme befähigen, Vorhersagen oder Entscheidungen zu treffen, ohne explizit dafür programmiert worden zu sein. Dieser Vorgang nennt sich \ac{ML}.
Die zu Verfügung gestellten Datensätze werden auch als Traingsdaten bezeichnet. Grunsätzlich bestehen sie aus Eingabeinformationen (Merkmalen) und Ausgabewerten (Labels oder Zielvariablen).
Besonders die Musterkennung im hochdimensionalen Raum, durch gleichzeitige Berücksichtiung von hunderten oder tausenden Merkmalen, macht das \ac{ML} außergewöhnlich leistungsfähig. Im vergleich dazu ist es für einen Menschen
schon schwierig drei- bis vierdimensionalle Sachverhalte zu erfassen~\footcite[\vglf][\pagef 5]{Matzka.2021}.

Es existieren unterschiedlichen Arten des \ac{ML}. Beim überwachten Lernen werden dem System, wie oben erwähnt, Trainigsdaten mit bekannten Eingaben und Ausgaben bereitgestellt. Daraus lernt 
das System, über eine Abbildungsfunktioniert, neue Eingaben für die Ausgaben abzubilden~\footcite[\vglf][\pagef 189]{Plaue.2021}. Bei der Methodtik des unüberwachten Lernens werden dem System nur Eingabedaten dargeboten und es wird erwartet,
dass es von selbst Muster und Strukturen in den Daten erkennt~\footcite[\vglf][\pagef 255]{Plaue.2021}. 
Das bestärkenden Lernen basiert auf der positiven oder negativen Rückmeldung auf eine bestimmte Aktion. 
Ziel ist es, dass das \ac{KI}-System auf der Grundlage der gemachten \enquote{Lernerfahrung} selbständig Vorhersagen und Entscheidungen treffen. 
Die Qualität dieser sind abhänging von der Qualität und Repräsentativität der verwendeten Daten. Auch muss der Mensch hier immernoch evaluieren, ob die getroffenen Vorhersagen oder Entscheidungen
zuverlässig und vertrauenswürdig sind.

\ac{KI}-Systeme mit \ac{ML} werden inbesondere in Bereichen mit Aufgabengebieten eingesetzt, wo Menschen schwierigkeiten haben diese zu lösen. Die menschliche Intelligenz wird dabei nicht ersetzt,
oder simuliert, sondern komplementiert~\footcite[\vglf][\pagef 5]{Matzka.2021}.


\subsection{Big Data}

Damit {KI}-Systeme lernen können, brauchen sie sehr große Datenmengen. Diese werden als Big Data bezeichnet. dabei handelt es sich um großen Datenmengen die in unterschiedlichen Formaten auftreten
und in verschiedenen Quellen generiert wurden. Der Author Ralf Huss definiert Big Data als Datenmengen, die zu groß, zu komplex oder zu schwach strukturiert sind, oder sich zu schnell ändern, um mit herkömmlichen
Methoden analysiert zu werden~\footcite[][\pagef 60]{Huss.2019}. Daran liegt die große Bedeutung von Big Data, wertvolle Erkenntnnisse und Muster aus den Daten zu extrahieren,
bei denen herkömmliche Analysemethoden nicht ausreichen würden.
Die Daten können dabei aus traditionellen Datenbanksystemen stammen, oder in unstrukturieten Formaten wie Text, Audio, Video, Sensordaten vorliegen.

Big Data besitzt drei Hauptcharakteristika. 

\begin{itemize}
    \item Volumen; Der Datenbestand bei Big Data kann enorme Ausmaße annehmen und liegt im Tera- (10\textsuperscript{12} Bytes) bis Zettabytebereich (10\textsuperscript{21} Bytes). 2008 wurden weltweit 10 Zettabytes (10\textsubscript{21} Bytes) verarbeitet~\footcite[\vglf][\pagef 61]{Huss.2019}. 
    \item Variety; Der Begriff bedeutet Vielfalt. Er bedeutet, dass strukturiete Daten aus z.B. Datenbanken, semi-strukturiete Daten wie z.B Logdateien oder Sensordaten und unstrukturierte Daten wie z.B. Textdokumente, E-Mails und Multimediadateien, gespeichert~\footcite[\vglf][\pagef 6]{Fasel.2019}.
    \item Velocity; Der entstehenden Datenstrom (Data Stream) bei Big Data wird in Echtzeit, oder nahezu Echtzeit, geneiert und muss von entsprechend schnellen Erfassungs-, Verarbeitungs- und Analysemethoden
    ebenfalls in Echzeit erfasst und analysiert werden~\footcite[\vglf][]{Fasel.2019}.
\end{itemize}


In einigen Qullen werden noch weitere Charakteristika definiert. 

\begin{itemize}
    \item Value; Der Wert des Unternehmens soll gesteigert werden~\footcite[\vglf][]{Fasel.2019}. Dabei ist nicht unbedingt allein der monetäre Wert gemeint. In Bezug auf die Daten muss geklärt werden,
    welche Erkenntnisse aus Ihnen abgeleitet werden können, um für das verarbeitende Unternemhem einen Mehrwert darzustellen. 
    \item Veracity; Da die Qualtität der Daten nicht per se bekannt ist, müssen spezielle Algorithmen eingesetzt werden, um die Qualität der Resultate .bzw die Plausbilität dieser zu evaluieren. Dabei garantiert
 ein grö0ere Datensatz keine bessere Aussagequalität~\footcite[\vglf][]{Fasel.2019}.
\end{itemize}

Die Herausforderungen bei Big Data umfassen vorallem die Datenerfassung, -speicherung, -verarbeitung und -analyse in angemessener Zeit, Datenschutz und Datensicherheit und insbesondere die Gewährleistung der 
Datenqulität. Um diese Herausforderungen zu bewältigen, müssen Technologien wie NoSQL-Datenbanken, Cloud-Computing und verteilte System eingesetzt werden.

Big Data hat das Potenzial, einen erheblichen Mehrwert für Unternehmen, Organisationen, Forschungseinrichtungen und die Gesellschaft insgesamt zu schaffen, indem es Einblicke und Erkenntnisse liefert, 
die zuvor nicht möglich waren. Es ermöglicht es, bessere Entscheidungen zu treffen, Effizienz und Produktivität zu steigern und Innovationen voranzutreiben.

\subsection{Datenschutzgrundverordnung}

Die \ac{DSGVO} wurde in ihrer jetzigen Form 2018 von der Europäischen Union verabschiedet und ist ein einheitlicher Rechtsrahmen mit dem ein verantwortungsbewusster Umgang
mit den personenbezogenen Daten der Bürger der \ac{EU} sicherstellt~\footcite[\vglf][\pagef 2]{Voigt.2018}.
Die Verordnung stärkt vor allem die Rechte der Bürger bei der Verarbeitung ihrer personenbezogenen Daten z.B. durch Unternehmen. Personenbezogene Daten sind alle Daten, die einen Menschen
\enquote{identifizierbar} machen. Die bloße Möglichkeit der \enquote{Identifizierung} durch eine Kombination verschiedener Informationen,die für sich alleine keinen Rückschluss 
auf den Betroffenen möglich gemacht hätte, aber es ermöglichen würde, reicht dabei aus um als personenbezogene Daten qualifiziert zu werden~\footcite[\vglf][\pagef 14]{Voigt.2018}

Im folgenden werden die wichtigsten Punkte der \ac{DSGVO} aufgeführt. Jeder \ac{EU}-Bürger hat das Recht zu erfahren, welche Daten ein Unternehmen über Ihn gesammelt hat, warum dieses Unternehmen
diese Daten sammelt, wie es sie verwendet und an wenn diese Daten übermittelt werden. Dies wird auch als Auskunfstrecht bezeichet. Weiterführend können diese Daten durch den Bürger
berichtigt werden, falls diese falsch oder unvollständig sind. Er hat das Recht der Berichtigung, aber auch das Recht seine Daten löschen zu lassen. Ebenfalls ist es ihn möglich die 
Verarbeitung durch das Unternehmen einzuschränken, oder dieser im gesamten zu wiedersprechen~\footcite[\vglf][\pagef 200]{Voigt.2018}. Des weiterhin hat er das Recht der Datenübertragbarbeit. Dabei müssen die Daten der
betroffenen Person in einem gängigen maschinenlesbaren Format übermittelt werden, oder diese einem anderen Unternehmen bereitstellen.
Personenbezogene Daten dürfen nicht ohne die Einwilligung der betroffenen Person erhoben oder verarbeitet werden. Dabei muss die Einwillgung freiwillig, spezifisch, informiert und
unmissverständlich sein.
Unternehmen müssen Datenpannen, zum Beispiel die offenlegung von personenbezogenen Daten, innerhalb von 72 Stunden an eine Datenschutzbehörde melden~\footcite[\vglf][\pagef 86]{Voigt.2018}.
Die \ac{DSGVO} ist noch deutlich umfangreicher und hat beträchtliche Auswirkung auf Unternemhen, besonders solche, die große Mengen an personenbezogenen Daten sammeln und verarbeiten.
Sie dient vorallem dem dem Schutz der Privatspähere der \ac{EU}-Bürger.
Verstöße gegen die DSGVO können zu erheblichen Strafen führen, und Unternehmen sind daher angehalten, ihre Datenverarbeitungsprozesse sorgfältig zu prüfen und zu verwalten.