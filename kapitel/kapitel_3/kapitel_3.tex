\newpage
\section{Zukünftige Herausforderungen}\label{Herausforderungen}

Mit dem Einsatz von \ac{KI} entstehen viele Herausforderungen für die Gesellschaft. Sie hat bereits und wird immer mehr Einfluss auf soziale, politische und ökonomische Systeme haben.
Viele Wissenschaftler befürchten, dass der Mensch sich in nahen Zukunft einen Wettlauf mit einer \enquote{Superintelligenz} liefern wird und diesen verlieren wird,
da Maschinen und \ac{KI} im speziellen viel schneller adaptiert und reagiert und der Mensch aus evolutionärer Sicht nicht mithalten kann. Mit diesem Hintergrund ist 
Notwendigkeit einer Außereinandersetzung mit den ethischen Dimensionen von \ac{KI} entstanden~\footcite[\vglf][\pagef 239]{Wittpahl.2018} und weiter anhaltend, da \ac{KI}s
immer tiefer in gesellschaftliche Thematiken vordringen und keinerlei Beschränkungen unterliegen. Das verändern von Wertschöpfungsprozessen, privater Kommunikation und 
besonders der zwischenmenschlichen Interaktion erfordert ethische Grenzen und Prinzipien. Der Einfluss der \ac{KI} wird noch deutlicher wachsen, 
wenn eine sich selbstgestaltende und fortentwickelnde, eine sog. starke \ac{KI}, Realität wird. Es werden dringend die Antworten auf die Fragen gesucht, was \ac{KI} mit
unserer Gesellschaft macht und wie sie unser momentanes Leben und Arbeitswelt weiter verändert wird~\footcite[\vglf][\pagef 239]{Wittpahl.2018}.

Algoithmen, die große Mengen an Daten analyiseren und nach Mustern suchen, eigenen sich sehr gut den Menschen zu erforschen. Die Daten, welche die intelligenten System von den 
Nutzern benötigen, um Ihnen das Leben zu erreichen, machen die Nutzer Zeitgleich durchsichtig. Unternehmen verkaufen diese Daten. Der Mensch wird zum Produkt~\footcite[\vglf][\pagef 114]{Lenzen.2020}.
Zum Schutz dieser Daten und der Privatsphäre wurde die \ac{DSGVO} in der \ac{EU} verabschiedet. Weitergehend müssen Mechanismen entworfen werden, die sicherstellen, dass
\ac{KI}-Systeme sicher und vertrauenswürdig sind. 
Denn diese System können auch zu Verzesserung und Vorurteilen führen, abhänging von den Ihren vorliegenden Daten. Kein Mensch darf durch \ac{KI}-Systeme diskriminiert werden.
Allerdings ist dies äußerst diffizil, da \ac{KI}-Systeme äußerst komplexe Entscheiden treffen, die für Menschen schwer zu verstehn und zu erklären sind. Es fehlt an 
Erklärbeit und Transparenz. Dies ist besonders priker in Bereich in dennen es um Menschenleben geht, wie z. B. in der Medizin. Die Entscheidungen müssen für den Menschen 
nachvollziehbar sein, nach welchen Kriterien diese entstanden sind~\footcite[\vglf][\pagef 240]{Wittpahl.2018}.
Daraus resultiert die Fragestellung, inwieweit die Menscheit den \ac{KI}-System die autonome Entscheidungsmöglichkeit überlässt. Es müssen Regularieren evaluiert werden,
die klären wer für die Entscheidungen eines \ac{KI}-System die Verantwortung übernimmt, z. B: auch beim autonomen Fahren im Falle eines Unfalls. Wie oben bereits erwähnt müssen ehtische Rahmenbedingungen geschaffen werden, 
die die Autonomie von \ac{KI}System regelt und eine klare Verantwortun für Ihre Entscheidungen und Handlungen festlegen~\footcite[\vglf][\pagef 35]{Robot.2023}.

Ein möglicher weg ist der der Aufbau von Transparenz und Überwachungsstrukturen, von Standards und Sanktionsmustern. Dies wird von vielen Wissenschaftlern als Grundvorrausetzung
für einen ethische verantwortungsvolle Nutzung angesehen~\footcite[\vglf][\pagef 243]{Wittpahl.2018}. Für den Umgang mit \ac{KI} braucht es Vertrauen, welche nur durch den
Souveränität und Kompetenz erreicht werden kann. Ein erster SChritt wurde dahingehnd schon unternommen. Initiativen wie AINOW oder OpenAI versuchen globale Standards für \ac{KI}-Systeme
etablieren um sie zu demokratisieren und für alle Menschen zugänglich zu machen. Große Unternehmen wie Google, Apple, Facebook und Amazon, stellen Ihre KI-Tools als 
Open-Source zur Verfügung, um die KI-Souveränität zu fördern und im gleichen Zuge von der Schwarmintelligenz zu profitieren. In Europa wird über eine spezielle
Forschungseinrictung für KI nachgedacht, an dennen das Wissen und die Kompetenzen zentral gebündelt werden~\footcite[\vglf][\pagef 243]{Wittpahl.2018}.

Daran ist gut gelegen, denn einfach den Stecker zu ziehen, ist nicht mehr Möglich. Die \ac{KI} muss nach ihren eigenen, von Menschen geschaffenen, 
Ethik handelnu~\footcite[\vglf][\pagef 244]{Wittpahl.2018} und der Mensch muss ihr vertrauen können.

% irgendwo einfügen ggf

Sobald \ac{KI} ein Teil unserer Gesellschaft ist, kann sie diese auch Beeinflußen, im positiven wie im negativen Sinne z. B. droht ein dringender Handlungsbedarf
im niedrig Qualifizierten Arbeitssektor. Viele Menschen könnten Ihren Arbeitsplatz verlieren. Die Menschheit muss sich neue Wege und Möglichkeiten suchen Ihr Leben zu 
gehalten. \ac{KI} wird das Leben, wie wir es heutzutage führen, stark verändern. Es ist vorstellbar, das Menschen Ihren Alltag ohne Arbeit gestalten werden. Dies ist eine
überaus umfangreiche Herausforderung.

% S. 265 für Ausblick dann ggf.
