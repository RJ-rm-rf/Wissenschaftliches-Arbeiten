\newpage
\section{Zukünftige Herausforderungen}\label{Herausforderungen}

Mit dem Einsatz von \ac{KI} entstehen viele Herausforderungen für die Gesellschaft. Sie hat bereits und wird immer mehr Einfluss auf soziale, politische und ökonomische Systeme nehmen.
Viele Wissenschaftler befürchten, dass der Mensch sich in naher Zukunft einen Wettlauf mit einer \enquote{Superintelligenz} liefern und diesen verlieren wird,
da Maschinen und die \ac{KI} im speziellen viel schneller adaptiert und reagiert und der Mensch aus evolutionärer Sicht nicht mithalten kann. Vor diesem Hintergrund ist die
Notwendigkeit einer Auseinandersetzung mit den ethischen Dimensionen der \ac{KI} entstanden~\footcite[\vglf][\pagef 239]{Wittpahl.2018} und weiter anhaltend, da \ac{KI}s
immer tiefer in gesellschaftliche Thematiken vordringen und keinerlei Beschränkungen unterliegen. Das verändern von Wertschöpfungsprozessen, privater Kommunikation und 
besonders der zwischenmenschlichen Interaktion erfordert ethische Grenzen und Prinzipien. Der Einfluss der \ac{KI} wird noch deutlicher wachsen, 
wenn eine sich selbstgestaltende und fortentwickelnde, eine sog. starke \ac{KI} Realität wird. Es werden dringend die Antworten auf die Fragen gesucht, was \ac{KI} mit
der Gesellschaft machen wird und wie sie das momentane Leben und die Arbeitswelt verändern wird~\footcite[\vglf][\pagef 239]{Wittpahl.2018}.

Algorithmen, die große Mengen an Daten analyiseren und nach Mustern suchen, eignen sich besonders gut, um den Menschen zu erforschen. Die Daten, welche die intelligenten Systeme von den 
Nutzern benötigen, um ihnen das Leben zu erleichtern, machen die Nutzer zeitgleich durchsichtig. Unternehmen nutzen diese Daten und profitieren von ihnen, indem sie diese verkaufen. Der Mensch wird somit zu einem Produkt~\footcite[\vglf][\pagef 114]{Lenzen.2020}.
Zum Schutz dieser Daten und der Privatsphäre wurde zwar die \ac{DSGVO} in der \ac{EU} verabschiedet, weitergehend müssen jedoch Mechanismen entworfen werden, welche sicherstellen, dass
\ac{KI}-Systeme sicher und vertrauenswürdig sind. 
Denn diese System können auf Grundlage, der von ihr vorliegenden Daten, auch zu einer Verzerrung und zu Vorurteilen führen. Kein Mensch darf durch \ac{KI}-Systeme diskriminiert werden.
Allerdings ist dies äußerst diffizil, da \ac{KI}-Systeme äußerst komplexe Entscheidungen treffen, die für Menschen schwer zu verstehen und zu erklären sind. Es fehlt somit an 
Erklärbeit und Transparenz. Dies ist besonders prekär in Bereichen, in denen es um das Leben von Menschen geht, wie z. B. in der Medizin. Die Entscheidungen müssen für den Menschen 
nachvollziehbar sein, nach welchen Kriterien diese entstanden sind~\footcite[\vglf][\pagef 240]{Wittpahl.2018}.
Daraus resultiert die Fragestellung, inwieweit die Menscheit den \ac{KI}-System die autonome Entscheidungsmöglichkeit überlassen. Es müssen Regularieren evaluiert werden,
die klären wer für die Entscheidungen eines \ac{KI}-System die Verantwortung übernimmt, wie z. B. beim autonomen Fahren im Falle eines Unfalls. Wie oben bereits erwähnt, müssen ethische Rahmenbedingungen geschaffen werden, 
welche die Autonomie von \ac{KI}System regeln und eine klare Verantwortung für Ihre Entscheidungen und Handlungen festlegen~\footcite[\vglf][\pagef 35]{Robot.2023}.

Ein möglicher Weg ist der Aufbau von Transparenz und Überwachungsstrukturen, von Standards und Sanktionsmustern. Dies wird von vielen Wissenschaftlern als Grundvorrausetzung
für eine ethische verantwortungsvolle Nutzung angesehen~\footcite[\vglf][\pagef 243]{Wittpahl.2018}. Für den Umgang mit \ac{KI} braucht es Vertrauen, welche nur durch die
Souveränität und Kompetenz erreicht werden kann. Ein erster SChritt wurde dahingehend schon unternommen, nämlich durch Initiativen wie AINOW oder OpenAI. Diese Unternehmen versuchen globale Standards für \ac{KI}-Systeme
etablieren, um sie zu demokratisieren und für alle Menschen zugänglich zu machen. Große Unternehmen wie Google, Apple, Facebook und Amazon, stellen ihre KI-Tools als 
Open-Source zur Verfügung, um die KI-Souveränität zu fördern und im gleichen Zuge von der Schwarmintelligenz zu profitieren. In Europa wird über eine spezielle
Forschungseinrichtung für KI nachgedacht, an denen das Wissen und die Kompetenzen zentral gebündelt werden kann~\footcite[\vglf][\pagef 243]{Wittpahl.2018}.

Daran ist gut gelegen, denn einfach den Stecker zu ziehen, ist nicht mehr möglich. Die \ac{KI} muss nach ihrer eigenen, von Menschen geschaffenen, 
ethischen Maßstäben handeln~\footcite[\vglf][\pagef 244]{Wittpahl.2018} und der Mensch muss ihr vertrauen können.

% irgendwo einfügen ggf

Sobald die \ac{KI} ein Teil unserer Gesellschaft ist, kann sie diese auch beeinflussen, im positiven als auch im negativen Sinne. Bspw. droht ein dringender Handlungsbedarf
im niedrig qualifizierten Arbeitssektor. Viele Menschen könnten Ihren Arbeitsplatz verlieren. Die Menschheit muss sich neue Wege und Möglichkeiten suchen ihr Leben zu 
gestalten. Die \ac{KI} wird das Leben, wie wir es heutzutage führen, stark verändern. Es ist vorstellbar, das Menschen ihren Alltag ohne Arbeit gestalten werden. Dies ist eine
überaus umfangreiche Herausforderung.

% S. 265 für Ausblick dann ggf.
